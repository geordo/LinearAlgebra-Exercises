\documentclass[12pt]{report}
\usepackage[margin = 1in]{geometry}

\usepackage[utf8]{vietnam}
\usepackage{fontspec}
\usepackage{amsmath}
\usepackage{graphicx}
\usepackage{lastpage}
\usepackage{hyperref}
\usepackage{fancyhdr} % Header and footer formatting

\usepackage{unicode-math}

% https://github.com/EntropyIncreaser/RTFG-Exercises
\setmathfont
	[Extension = .otf,
	math-style= TeX,
	BoldFont = XITSMath-Bold.otf,
	BoldItalicFont = XITS-BoldItalic.otf
]{XITSMath-Regular}

\setmainfont{Times New Roman}

\hypersetup{
	colorlinks = true,
	linkcolor = blue,
	citecolor = red,
	urlcolor = teal}

\pagestyle{fancy}
\renewcommand{\headrulewidth}{1pt}
\renewcommand{\footrulewidth}{1pt}
\setlength{\headheight}{20pt}

% Header and footer information
\lhead{\emph{Đậu Đức Đạt}}
\chead{}
\rhead{Toán Cao Cấp Đại Số Tuyến Tính}
\lfoot{}
%\cfoot{Page \thepage\ of \ \pageref{LastPage}}
\cfoot{Trang \thepage}
\rfoot{}

\begin{document}

\section*{Chương 2: Ma trận - Định thức}

\textbf{Bài 1.} 
Cho $A = \begin{bmatrix}
	1 & 3 \\ -1 & 2 \\ 3 & 4
	\end{bmatrix}, B = \begin{bmatrix} 
	0 & 3 & -2 \\ 1 & 2 & 3 \end{bmatrix}$. 
Tính $A - 3B^T$.\\
\textbf{Bài giải.}
 $A - 3B^T = \begin{bmatrix}
	1 & 3 \\ -1 & 2 \\ 3 & 4 
	\end{bmatrix} - 3\begin{bmatrix} 
	0 & 1 \\ 3 & 2 \\ -2 & 3 
	\end{bmatrix} = \begin{bmatrix}
	1 & 0 \\ -10 & -4 \\ 9 & -5 
	\end{bmatrix}$\\ \\
\textbf{Bài 2.} Các ma trận sau, ma trận nào nhân được với nhau. Hãy nhân các ma trận nếu được\\ 
$A = \begin{bmatrix} 1 & 1 & - 1 \\ 2 & -1 & 3 \end{bmatrix}; 
	B = \begin{bmatrix} 3 & 2 & 1 \\ 1 & 1 & 0 \end{bmatrix};
	C = \begin{bmatrix} 1 & 1 \\ 1 & 2 \\ 2 & 3 \end{bmatrix};
	D = \begin{bmatrix} -2 \\ 1 \\ 3 \end{bmatrix}$\\ \\
\textbf{Bài giải.}\\
\textbf{Quy tắc:} $A_{m,q}B_{q,n}=C_{m,n}$ (Số cột ma trận $A$ = số hàng ma trận $B$)\\
\textbf{Lưu ý:} Phép nhân ma trận không có tính chất hoán vị nên $AB \neq BA$\\
$AC = \begin{bmatrix} 1 & 1 & - 1 \\ 2 & -1 & 3 \end{bmatrix} \begin{bmatrix} 1 & 1 \\ 1 & 2 \\ 2 & 3 \end{bmatrix} = \begin{bmatrix} 0 & 0 \\ 7 & 9\end{bmatrix}\\$
$AD = \begin{bmatrix} 1 & 1 & - 1 \\ 2 & -1 & 3 \end{bmatrix} \begin{bmatrix} -2 \\ 1 \\ 3 \end{bmatrix} = \begin{bmatrix} -4 \\ 4\end{bmatrix}\\$
$BC = \begin{bmatrix} 3 & 2 & 1 \\ 1 & 1 & 0 \end{bmatrix} \begin{bmatrix} 1 & 1 \\ 1 & 2 \\ 2 & 3 \end{bmatrix} = \begin{bmatrix} 7 & 10 \\ 2 & 3 \end{bmatrix}\\$
$BD = \begin{bmatrix} 3 & 2 & 1 \\ 1 & 1 & 0 \end{bmatrix} \begin{bmatrix} -2 \\ 1 \\ 3 \end{bmatrix} = \begin{bmatrix} -1 \\ -1 \end{bmatrix}\\$
$CA = \begin{bmatrix} 1 & 1 \\ 1 & 2 \\ 2 & 3 \end{bmatrix} \begin{bmatrix} 1 & 1 & - 1 \\ 2 & -1 & 3 \end{bmatrix} = \begin{bmatrix} 3 & 0 & 2 \\ 5 & -1 & 5 \\ 8 & -1 & 7 \end{bmatrix}\\$
$CB = \begin{bmatrix} 1 & 1 \\ 1 & 2 \\ 2 & 3 \end{bmatrix} \begin{bmatrix} 3 & 2 & 1 \\ 1 & 1 & 0 \end{bmatrix} = \begin{bmatrix} 4 & 3 & 1 \\ 5 & 4 & 1 \\ 9 & 7 & 2 \end{bmatrix}\\$\\
\\ \textbf{Bài 3.}Tìm $f(A)$ với $f(x) = x^2 - 5x + 3$ và $A = \begin{bmatrix} 1 & -1 \\ -3 & 3 \end{bmatrix}$\\
\\ \textbf{Bài 4.} Cho $A = \begin{bmatrix} a & 1 & 0 \\ 0 & a & 1 \\ 0 & 0 & a \end{bmatrix}$. Tìm $A^{1000}$.\\
\\ \textbf{Bài 5.} Cho $A = \begin{bmatrix} 2 & 1 & 1 \\ 3 & 1 & 0 \\ 0 & 1 & 2 \end{bmatrix}$. Tìm $A^2, A^{-1}, A^{-2}$.\\
\\ \textbf{Bài 6.} Tính các định thức\\
$a)\begin{vmatrix} 0 & 1 & 1 \\ 1 & 0 & 1 \\ 1 & 1 & 0 \end{vmatrix}$\\
$b)\begin{vmatrix} 1 & 1 & 1 \\ 1 & 2 & 3 \\ 1 & 3 & 6 \end{vmatrix}$\\
$c)\begin{vmatrix} 3 & 1 & 1 & 1 \\ 1 & 3 & 1 & 1 \\ 1 & 1 & 3 & 1 \\ 1 & 1 & 1 & 3 \end{vmatrix}$\\
$d)\begin{vmatrix} 1 & 2 & 3 & 4 \\ 2 & 3 & 4 & 1 \\ 3 & 4 & 1 & 2 \\ 4 & 1 & 2 & 3 \end{vmatrix}$\\
$e)\begin{vmatrix} a+b & c & 1 \\ b+c & a & 1 \\ c+a & b & 1 \end{vmatrix}$\\
$f)\begin{vmatrix} 246 & 427 & 327 \\ 1014 & 543 & 441 \\ -342 & 721 & 621 \end{vmatrix}$\\
$g)\begin{vmatrix} 3 & 4 & 0.347 & 7 \\ 6 & 2 & 0.628 & 8 \\ 4 & 9 & 0.491 & 1 \\ 9 & 4 & 0.942 & 2 \end{vmatrix}$\\
\\ \textbf{Bài 7.} Tính các định thức cấp n\\
$a) \begin{vmatrix} 
	a & x & \cdots & x \\ 
	x & a & \cdots & x \\ 
	\vdots & \vdots & \ddots & \vdots \\
	x & x & \cdots & a
\end{vmatrix}\\
b) \begin{vmatrix} 
	1 + a_1 & a_2 & \cdots & a_n \\ 
	a_1 & 1 + a_2 & \cdots & a_n \\ 
	\vdots & \vdots & \ddots & \vdots \\
	a_1 & a_2 & \cdots & 1+a_n
\end{vmatrix}$\\
\\ \textbf{Bài 8.} Không tính định thức, chứng minh rằng:\\
$a)\ \begin{vmatrix}
	y+z & z+x & x+y \\
	y_1+z_1 & z_1+x_1 & x_1+y_1 \\
	y_2+z_2 & z_2+x_2 & x_2+y_2
\end{vmatrix}
= 2\begin{vmatrix}
	x & y & z \\
	x_1 & y_1 & z_1 \\
	x_2 & y_2 & z_2
\end{vmatrix}$\\
$b)\ \begin{vmatrix}
	1 & a & a^3 \\
	1 & b & b^3 \\
	1 & c & c^3
\end{vmatrix}
= (b-a)(c-a)(c-b)(a+b+c)$\\
$c)\ \begin{vmatrix}
	a_1+b_1x & a_1x+b_1 & c_1 \\
	a_2+b_2x & a_2x+b_2 & c_2 \\
	a_3+b_3x & a_3x+b_3 & c_3
\end{vmatrix}
= (1-x^2)
\begin{vmatrix}
	a_1 & b_1 & c_1 \\
	a_2 & b_2 & c_2 \\
	a_3 & b_3 & c_3
\end{vmatrix}$

\end{document}
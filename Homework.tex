\documentclass[12pt]{report}
% Can le
\usepackage[left=2cm,right=2cm,top=2cm,bottom=2cm]{geometry}

\usepackage[utf8]{vietnam}
\usepackage{fontspec}
\usepackage{amsmath}
\usepackage{graphicx}
\usepackage{lastpage}
\usepackage{hyperref}
\usepackage{fancyhdr} % Header and footer formatting

\usepackage{unicode-math}

% Tat thut vao dau dong moi doan
\setlength{\parindent}{0pt}
% Khoang cach giua 2 doan
\setlength{\parskip}{0.8em}
%\renewcommand{\baselinestretch}{1.5}

% https://github.com/EntropyIncreaser/RTFG-Exercises
\setmathfont
[Extension = .otf,
math-style= TeX,
BoldFont = XITSMath-Bold.otf,
BoldItalicFont = XITS-BoldItalic.otf
]{XITSMath-Regular}

\setmainfont{Times New Roman}

\hypersetup{
	colorlinks = true,
	linkcolor = blue,
	citecolor = red,
	urlcolor = teal}

\pagestyle{fancy}
\renewcommand{\headrulewidth}{1pt}
\renewcommand{\footrulewidth}{1pt}
\setlength{\headheight}{20pt}

% Header and footer information
\lhead{\textit{Đậu Đức Đạt}}
\chead{}
\rhead{}
\lfoot{}
%\cfoot{Page \thepage\ of \ \pageref{LastPage}}
\cfoot{Trang \thepage}
\rfoot{}

\begin{document}
	
% {\huge \textbf{Bài tập về nhà}}

{\large \textbf{I. Kiến thức cơ bản}}

\textbf{1. Phương pháp truy hồi}

Tìm một hệ thức giữa định thức cấp $n$ và các định thức cấp thấp hơn được định nghĩa tương tự. Trường hợp hay gặp nhất là dạng 
\[ D_n = pD_{n-1}+qD_{n-2} \]
\begin{itemize}
	\item Nếu $q=0$ thì $D_n = p^{n-1}D_1$
	\item Nếu $q \neq 0$ thì ta gọi $\alpha$ và $\beta$ là 2 nghiệm của phương trình $x^2-px-q = 0$
	\begin{itemize}
		\item Nếu $\alpha \neq \beta$ thì 
		\[ D_n = \frac{D_2 - \beta D_1}{\alpha (\alpha-\beta)} \alpha^n + \frac{D_2 - \alpha D_1}{\beta (\beta -\alpha)} \beta^n \]
		\item Nếu $\alpha = \beta$ thì 
		\[ D_n = (n-1)\alpha^{n-2}D_2 - (n-2)\alpha^{n-1}D_1 \]
	\end{itemize}
\end{itemize}

\textbf{2. Công thức khai triển nhị thức Newton}

{\large \textbf{II. Bài tập}}

\textbf{Bài 10.} Tính các định thức sau

a) $\begin{vmatrix}
a+b & a-b \\
a-b & a+b
\end{vmatrix} = (a+b)^2 - (a-b)^2 = 4ab$

b) $\begin{vmatrix}
\cos \varphi & -\sin \varphi \\
\sin \varphi & \cos \varphi
\end{vmatrix} = \cos ^ 2 \varphi + \sin ^ 2 \varphi = 1$

c) $\begin{vmatrix}
a+ ib & b \\
2a & a-ib
\end{vmatrix} = (a+ib)(a-ib) - 2ab = a^2 + b^2 - 2ab = (a-b)^2$ 

d) $\begin{vmatrix}
1 & 2 & 3 \\
4 & 5 & 6 \\
7 & 8 & 9
\end{vmatrix} = \begin{vmatrix}
1 & 1 & 2 \\
4 & 1 & 2 \\
7 & 1 & 2
\end{vmatrix} \left( \begin{smallmatrix}
-c_1 + c_2 &\to c_2 \\
-2c_1 + c_3 &\to c_3
\end{smallmatrix}\right) = 0$

e) $\begin{vmatrix}
a+x & x & x \\
x & b+x & x \\
x & x & c+x
\end{vmatrix} = (a+x)(b+x)(c+x) + 2x^3 - x^2(b+x) - x^2(c+x) - x^2(a+x) = abc + x(ab + ac + bc)$

f) $\begin{vmatrix}
\sin \alpha & \cos \alpha & 1 \\
\sin \beta & \cos \beta & 1 \\
\sin \gamma & \cos \gamma & 1 \\
\end{vmatrix} =\sin\alpha\cos\beta+\sin\beta\cos\gamma+\sin\gamma\cos\alpha - \cos\beta\sin\gamma-\cos\gamma\sin\alpha-\cos\alpha\sin\beta$

$= \sin(\alpha - \beta) + \sin(\beta - \gamma) + \sin(\gamma - \alpha) = 2\sin \frac{\alpha - \gamma}{2} \cos \frac{\alpha - 2\beta + \gamma}{2} + 2\sin \frac{\gamma - \alpha}{2} \cos \frac{\gamma - \alpha}{2}$

$= 2 \sin \frac{\alpha - \gamma}{2}(\cos \frac{\alpha - 2\beta + \gamma}{2} - \cos \frac{\gamma - \alpha}{2}) = -4\sin \frac{\alpha - \beta}{2}\sin \frac{\beta-\gamma}{2}\sin \frac{\gamma-\alpha}{2}$

g) $\begin{vmatrix}
	1 & 1 & 1 \\
	1 & \varepsilon & \varepsilon^2 \\
	1 & \varepsilon^2 & \varepsilon
\end{vmatrix}$ với $\varepsilon = \cos \frac{4\pi}{3} + i\sin \frac{4\pi}{3}$

\textbf{Bài 11.} Giải các phương trình và bất phương trình sau: 

a) $\begin{vmatrix}
	x & x+1 & x+2 \\
	x+3 & x+4 & x+5 \\
	x+6 & x+7 & x+8
\end{vmatrix} = 0$

\textbf{Bài giải.}  

$\begin{vmatrix}
	x & x+1 & x+2 \\
	x+3 & x+4 & x+5 \\
	x+6 & x+7 & x+8
\end{vmatrix} = 0 \Leftrightarrow \begin{vmatrix}
x & 1 & 2 \\
x+3 & 1 & 2 \\
x+6 & 1 & 2
\end{vmatrix} \left( \begin{smallmatrix} -c_1 + c_2 \to c_2 \\ -c_1+c_3 \to c_3 \end{smallmatrix} \right) = 0$

Do $\begin{vmatrix}
	x & 1 & 2 \\
	x+3 & 1 & 2 \\
	x+6 & 1 & 2
\end{vmatrix} = 0\ \forall x \in R$, vậy nghiệm của phương trình là $R$.

b) $\begin{vmatrix}
	2 & x+2 & -1 \\
	1 & 1 & -2 \\
	5 & -3 & x
\end{vmatrix} \geq 0$

\textbf{Bài giải.} 

$\begin{vmatrix}
	2 & x+2 & -1 \\
	1 & 1 & -2 \\
	5 & -3 & x
\end{vmatrix} \geq 0 \Leftrightarrow 2x+3-10(x+2)+5-12-x(x+2) \geq 0$

$\Leftrightarrow -x^2 -10x - 24 \geq 0$ 

$\Leftrightarrow -6 \leq x \leq -4$ 

\textbf{Bài 12.} Chứng minh rằng: 

a) $\begin{vmatrix}
	a_1+b_1x & a_1x+b_1 & c_1 \\
	a_2+b_2x & a_2x+b_2 & c_2 \\
	a_3+b_3x & a_3x+b_3 & c_3 \\
\end{vmatrix} = (1-x^2) \begin{vmatrix}
	a_1 & b_1 & c_1 \\ 
	a_2 & b_2 & c_2 \\
	a_3 & b_3 & c_3 \\
\end{vmatrix}$

\textbf{Bài giải.} 

$\begin{vmatrix}
	a_1+b_1x & a_1x+b_1 & c_1 \\
	a_2+b_2x & a_2x+b_2 & c_2 \\
	a_3+b_3x & a_3x+b_3 & c_3 \\
\end{vmatrix} = (1+x) \begin{vmatrix}
a_1+b_1 & a_1x+b_1 & c_1 \\
a_2+b_2 & a_2x+b_2 & c_2 \\
a_3+b_3 & a_3x+b_3 & c_3 \\
\end{vmatrix} (c_2+c_1 \to c_1)$

$= (1+x)(x-1)\begin{vmatrix}
	a_1+b_1 & a_1 & c_1 \\
	a_2+b_2 & a_2 & c_2 \\
	a_3+b_3 & a_3 & c_3 \\
\end{vmatrix} (c_2 - c_1 \to c_2) = (1+x)(x-1) \begin{vmatrix}
b_1 & a_1 & c_1 \\ 
b_2 & a_2 & c_2 \\
b_3 & a_3 & c_3 \\
\end{vmatrix} (-c_2 +c_1 \to c_1)$

$= (1-x^2) \begin{vmatrix}
	a_1 & b_1 & c_1 \\ 
	a_2 & b_2 & c_2 \\
	a_3 & b_3 & c_3 \\
\end{vmatrix} (c_1 \Leftrightarrow c_2)$ \textbf{ĐPCM}

b) $\begin{vmatrix}
	1 & a & a^3 \\
	1 & b & b^3 \\
	1 & b & c^3 \\
\end{vmatrix} = (a+b+c) \begin{vmatrix}
	1 & a & a^2 \\
	1 & b & b^2 \\
	1 & b & c^2 \\
\end{vmatrix}$

\textbf{Bài giải.} 

$\begin{vmatrix}
	1 & a & a^3 \\
	1 & b & b^3 \\
	1 & b & c^3 \\
\end{vmatrix} = \begin{vmatrix}
1 & a & a^3 + a(ab+ac+bc) - abc \\
1 & b & b^3 + b(ab+ac+bc) - abc \\
1 & b & c^3 + c(ab+ac+bc) - abc \\
\end{vmatrix} \left( \begin{smallmatrix}
(-abc)c_1 + c_3 &\to c_3 \\ (ab+ac+bc)c_2 + c_3 &\to c_3
\end{smallmatrix} \right) = (a+b+c) \begin{vmatrix}
1 & a & a^2 \\
1 & b & b^2 \\
1 & b & c^2 \\
\end{vmatrix}$ \textbf{ĐPCM}.

\textbf{Bài 13.} Tính các định thức sau:

a) $\begin{vmatrix}
	1 & 1 & 1 \\ 
	x & y & z \\
	x^2 & y^2 & z^2 \\
\end{vmatrix}$

b) $\begin{vmatrix}
	x & y & x+y \\ 
	x & x+y & y \\
	x+y & y & y \\
\end{vmatrix}$

\textbf{Bài 14.} Vẽ đồ thị hàm số sau: $y = \frac{1}{b-a}\begin{vmatrix}
	x & x^2 & 1 \\ 
	a & a^2 & 1 \\
	b & b^2 & 1 \\
\end{vmatrix}, a \neq b$

\textbf{Bài 15.} Xác định dấu của các số hạng sau trong định thức có cấp tương ứng 

a) $a_{43}a_{21}a_{35}a_{12}a_{54}$

b) $a_{33}a_{16}a_{72}a_{27}a_{55}a_{61}a_{44}$

\textbf{Bài 16.} Tìm các số hạng của định thức

$\begin{vmatrix}
	5x & 1 & 2 & 3 \\
	x & x & 1 & 2 \\
	1 & 2 & x & 3 \\
	x & 1 & 2 & 2x \\
\end{vmatrix}$, chứa $x^3$ và $x^4$

\textbf{Bài 17.} Tính các định thức sau

a) $\begin{vmatrix}
	1 & 1 & 1 & 1 \\
	1 & -1 & 2 & 2 \\
	1 & 1 & -1 & 3 \\
	1 & 1 & 1 & -1 \\
\end{vmatrix}$

b) $\begin{vmatrix}
	2 & 1 & 1 & 1 & 1 \\
	1 & 3 & 1 & 1 & 1 \\
	1 & 1 & 4 & 1 & 1 \\
	1 & 1 & 1 & 5 & 1 \\
	1 & 1 & 1 & 1 & 6 \\
\end{vmatrix}$

c) $\begin{vmatrix}
0 & b & c & d \\
b & 0 & d & c \\
c & d & 0 & b \\
d & c & b & 0 \\
\end{vmatrix}$

\textbf{Cách 1.} Khai triển Laplace theo cột 1, ta có:

$det(A) = b.A_{2,1} + c.A_{3,1} + d.A_{4,1} = -b\begin{vmatrix}
b & c & d \\
d & 0 & b \\
c & b & 0
\end{vmatrix} + c\begin{vmatrix}
b & c & d \\
0 & d & c \\
c & b & 0
\end{vmatrix} - d\begin{vmatrix}
b & c & d \\
0 & d & c \\
d & 0 & b
\end{vmatrix}$

$= -b\left(c^2b+d^2b-b^3\right) + c\left(c^3-d^2c-b^2c\right) -d\left(b^2d+c^2d-d^3\right) = b^4+c^4+d^4-2b^2c^2-2c^2d^2-2d^2b^2$	

$= b^4+c^4-2b^2c^2-b^2d^2-c^2d^2-2bcd^2-b^2d^2-c^2d^2+2bcd^2+d^4 = \left(b^2-c^2\right)^2-\left(b+c\right)^2d^2-\left(b-c\right)^2d^2+d^4 = \left(b^2-c^2\right)^2d^2 - \left(b^2+c^2\right)^2d^2 - (b-c)^2d^2+d^4 = \lbrack (b+c)^2-d^2 \rbrack \lbrack (b-c)^2-d^2 \rbrack$

$= (b+c-d)(b+c+d)(b-c-d)(b-c+d)$

\textbf{Cách 2.} Đặt $P(x) = \begin{vmatrix}
-x & b & c & d \\
b & -x & d & c \\
c & d & -x & b \\
d & c & b & -x \\
\end{vmatrix}$, thì $P(x)$ là một đa thức bậc 4 có số hạng với lũy thừa cao nhất là $x^4$

$P(x) = \begin{vmatrix}
b+c+d-x & b+c+d-x & b+c+d-x & b+c+d-x \\
b & -x & d & c \\
c & d & -x & b \\
d & c & b & -x \\
\end{vmatrix} (h_2+h_3+h_4+h_1 \to h_1)$

$\Rightarrow P(b+c+d) = 0$

$P(x) = \begin{vmatrix}
	b-x & b-x & c+d & c+d \\
	b & -x & d & c \\
	c+d & c+d & b-x & b-x \\
	d & c & b & -x \\
\end{vmatrix} \left( \begin{smallmatrix}
h_2+h_1 \to h_1\\
h_4+h_3 \to h_3
\end{smallmatrix} \right)$

$\Rightarrow P(b-c-d) = 0$

$P(x) = \begin{vmatrix}
	c-x & b+d & c-x & b+d \\
	b+d & c-x & b+d & c-x \\
	c & c & -x & b \\
	d & c & b & -x \\
\end{vmatrix} \left( \begin{smallmatrix}
	h_3+h_1 \to h_1\\
	h_4+h_2 \to h_2
\end{smallmatrix} \right)$

$\Rightarrow P(c-b-d) = 0$

$P(x) = \begin{vmatrix}
	d-x & b+c & b+c & d-x \\
	b+c & d-x & d-x & b+c \\
	c & d & -x & b \\
	d & c & b & -x \\
\end{vmatrix} \left( \begin{smallmatrix}
	h_4+h_1 \to h_1\\
	h_3+h_2 \to h_2
\end{smallmatrix} \right)$

$\Rightarrow P(d-b-c) = 0$

Vậy $x=b+c+d; x=b-c-d; x=c-b-d; x=d-b-c$ là 4 nghiệm của $P(x)$.

Vậy $P(x)=(x-b-c-d)(x-b+c+d)(x+b-c+d)(x+b+c-d)$  

Trở lại bài toán, với $x = 0, P(0) = (-b-c-d)(-b+c+d)(b-c+d)(b+c-d)$

d) $\begin{vmatrix}
	5 & 6 & 0 & 0 & 0 \\
	1 & 5 & 6 & 0 & 0 \\
	0 & 1 & 5 & 6 & 0 \\
	0 & 0 & 1 & 5 & 6 \\
	0 & 0 & 0 & 1 & 5 \\
\end{vmatrix}$

\textbf{Bài giải.} Đặt $d_5 = \begin{vmatrix}
	5 & 6 & 0 & 0 & 0 \\
	1 & 5 & 6 & 0 & 0 \\
	0 & 1 & 5 & 6 & 0 \\
	0 & 0 & 1 & 5 & 6 \\
	0 & 0 & 0 & 1 & 5 \\
\end{vmatrix}$

Khai triển Laplace theo dòng 1, ta có: $d_5 = 5d_4 -6d_3; d_1 = 5, d_2 = 19$

Công thức tổng quát: $d_n = 3^{n+1} - 2^{n+1} \Rightarrow d_5 = 665$

e) $\begin{vmatrix}
	0 & -a & -b & -d \\
	a & 0 & -c & -e \\
	b & c & 0 & 0 \\
	d & e & 0 & 0 \\
\end{vmatrix}$

f) $\begin{vmatrix}
	2 & 1 & 1 & 0 & 0 & 0 \\
	2 & 3 & 4 & 0 & 0 & 0 \\
	3 & 6 & 10 & 0 & 0 & 0 \\
	4 & 9 & 14 & 1 & 1 & 1 \\
	5 & 15 & 24 & 1 & 5 & 9 \\
	0 & 24 & 38 & 1 & 25 & 81 \\
\end{vmatrix}$

g) $\begin{vmatrix}
	a & ab & 0 & \cdots & 0 & 0 \\
	1 & a+b & ab & \cdots & 0 & 0 \\
	0 & 1 & a+b & \cdots & 0 & 0 \\
	\vdots & \vdots & \vdots & \ddots & \vdots & \vdots \\
	0 & 0 & 0 & \cdots & a+b & ab \\
	0 & 0 & 0 & \cdots & 1 & a+b \\
\end{vmatrix}$

h) $\begin{vmatrix}
	a+b & ab & 0 & \cdots & 0 & 0 \\
	1 & a+b & ab & \cdots & 0 & 0 \\
	0 & 1 & a+b & \cdots & 0 & 0 \\
	\vdots & \vdots & \vdots & \ddots & \vdots & \vdots \\
	0 & 0 & 0 & \cdots & a+b & ab \\
	0 & 0 & 0 & \cdots & 1 & a+b \\
\end{vmatrix}$

\textbf{Bài giải.} Đặt $D_n = \begin{vmatrix}
	a+b & ab & 0 & \cdots & 0 & 0 \\
	1 & a+b & ab & \cdots & 0 & 0 \\
	0 & 1 & a+b & \cdots & 0 & 0 \\
	\vdots & \vdots & \vdots & \ddots & \vdots & \vdots \\
	0 & 0 & 0 & \cdots & a+b & ab \\
	0 & 0 & 0 & \cdots & 1 & a+b \\
\end{vmatrix}$

Khai triển Laplace theo dòng 1, ta có: $D_n = (a+b)D_{n-1} - ab D_{n-2}; D_1 = (a+b), D_2 = a^2+b^2+ab$

\begin{itemize}
	\item Nếu $ab = 0$ thì \[ D_n = (a+b)^n \]
	\item Nếu $ab \neq 0$ thì gọi $\alpha$ và $\beta$ là 2 nghiệm của phương trình $x^2 - (a+b)x + ab = 0$
	\begin{itemize}
		\item Nếu $\alpha = \beta \Leftrightarrow \Delta = 0 \Rightarrow \alpha = \beta = \frac{a+b}{2}$ 
		
		$\Rightarrow D_n = (n-1) \left( \frac{a+b}{2} \right)^{n-2} (a^2+b^2+ab) - (n-2) \left(\frac{a+b}{2}\right)^{n-1} (a+b) = \frac{(a+b)^{n-2}(2na^2+2nb^2+4ab)}{2^n}$
		
		\item Nếu $\alpha \neq \beta \Leftrightarrow \Delta > 0 \Rightarrow \alpha = a, \beta = b$
		
		$\Rightarrow D_n = \frac{a^2+b^2+ab - b(a+b)}{a(a-b)} a^n + \frac{a^2+b^2+ab - a(a+b)}{b(b-a)} b^n = \frac{a^{n+1} - b^{n+1}}{a-b}$
	\end{itemize}
\end{itemize}

i) $\begin{vmatrix}
	3 & 2 & 2 & \cdots & 2 & 2 \\
	2 & 3 & 2 & \cdots & 2 & 2 \\
	2 & 2 & 3 & \cdots & 2 & 2 \\
	\vdots & \vdots & \vdots & \ddots & \vdots & \vdots \\
	2 & 2 & 2 & \cdots & 3 & 2 \\
	2 & 2 & 2 & \cdots & 2 & 3 \\
\end{vmatrix}$

j) $\begin{vmatrix}
	2 & 1 & 0 & \cdots & 0 & 0 \\
	1 & 2 & 1 & \cdots & 0 & 0 \\
	0 & 1 & 2 & \cdots & 0 & 0 \\
	\vdots & \vdots & \vdots & \ddots & \vdots & \vdots \\
	0 & 0 & 0 & \cdots & 2 & 1 \\
	0 & 0 & 0 & \cdots & 1 & 2 \\
\end{vmatrix}$

k) $\begin{vmatrix}
	x & 1 & 0 & \cdots & 0 & 0 \\
	1 & x & 1 & \cdots & 0 & 0 \\
	0 & 1 & x & \cdots & 0 & 0 \\
	\vdots & \vdots & \vdots & \ddots & \vdots & \vdots \\
	0 & 0 & 0 & \cdots & x & 1 \\
	0 & 0 & 0 & \cdots & 1 & x \\
\end{vmatrix}$

l) $\begin{vmatrix}
	c & a & 0 & \cdots & 0 & 0 \\
	b & c & a & \cdots & 0 & 0 \\
	0 & b & c & \cdots & 0 & 0 \\
	\vdots & \vdots & \vdots & \ddots & \vdots & \vdots \\
	0 & 0 & 0 & \cdots & c & a \\
	0 & 0 & 0 & \cdots & b & c \\
\end{vmatrix}$

\textbf{Bài 18.} Cho $A = \left( a_{ij}\right) \in M_n(K)$ sao cho $a_{ij} = \overline{a_{ji}}$. Chứng minh rằng: $det(A) \in R$

\textbf{Bài 19.} Chứng minh rằng định thức cấp lẻ bằng không nếu tất cả các phần tử của nó thỏa mãn điều kiện: $a_{ij} + a_{ji} = 0$

\textbf{Bài 20.} Cho $A \in M_n(K)$. Chứng minh rằng $det \begin{pmatrix}
	A & A^2 \\ A^2 & A^3 
\end{pmatrix} = 0$

\textbf{Bài 21.} Chứng minh rằng nếu $A$ và $B$ giao hoán thì 
\[ det \begin{pmatrix} A & B \\ C & D \end{pmatrix} = det(DA - CB) \]

\textbf{Bài 22.} Chứng minh rằng nếu $A = (a_ij) \in M_n(K)$ và $B$ thoả điều kiện \[ a_{kk} > \displaystyle \sum_{\substack{s=1 \\ s \neq k}}^n |a_{ks}|, k = 1, 2, \cdots n \text{ thì } detA \neq 0 \]

Xin chào các bạn

\end{document}
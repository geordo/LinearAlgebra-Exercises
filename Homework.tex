\documentclass[12pt]{report}
% Can le
\usepackage[left=2cm,right=2cm,top=2cm,bottom=2cm]{geometry}

\usepackage[utf8]{vietnam}
\usepackage{fontspec}
\usepackage{amsmath}
\usepackage{graphicx}
\usepackage{lastpage}
\usepackage{hyperref}
\usepackage{fancyhdr} % Header and footer formatting

\usepackage{unicode-math}

% Tat thut vao dau dong moi doan
\setlength{\parindent}{0pt}
% Khoang cach giua 2 doan
\setlength{\parskip}{0.8em}
%\renewcommand{\baselinestretch}{1.5}

% https://github.com/EntropyIncreaser/RTFG-Exercises
\setmathfont
[Extension = .otf,
math-style= TeX,
BoldFont = XITSMath-Bold.otf,
BoldItalicFont = XITS-BoldItalic.otf
]{XITSMath-Regular}

\setmainfont{Times New Roman}

\hypersetup{
	colorlinks = true,
	linkcolor = blue,
	citecolor = red,
	urlcolor = teal}

\pagestyle{fancy}
\renewcommand{\headrulewidth}{1pt}
\renewcommand{\footrulewidth}{1pt}
\setlength{\headheight}{20pt}

% Header and footer information
\lhead{\textit{Đậu Đức Đạt}}
\chead{}
\rhead{}
\lfoot{}
%\cfoot{Page \thepage\ of \ \pageref{LastPage}}
\cfoot{Trang \thepage}
\rfoot{}

\begin{document}
	
{\huge \textbf{Bài tập về nhà}}

\textbf{Bài 10.} Tính các định thức sau

a) $\begin{vmatrix}
a+b & a-b \\
a-b & a+b
\end{vmatrix} = (a+b)^2 - (a-b)^2 = 4ab$

b) $\begin{vmatrix}
\cos \varphi & -\sin \varphi \\
\sin \varphi & \cos \varphi
\end{vmatrix} = \cos ^ 2 \varphi + \sin ^ 2 \varphi = 1$

c) $\begin{vmatrix}
a+ ib & b \\
2a & a-ib
\end{vmatrix} = (a+ib)(a-ib) - 2ab = a^2 + b^2 - 2ab = (a-b)^2$ 

d) $\begin{vmatrix}
1 & 2 & 3 \\
4 & 5 & 6 \\
7 & 8 & 9
\end{vmatrix} = \begin{vmatrix}
1 & 1 & 2 \\
4 & 1 & 2 \\
7 & 1 & 2
\end{vmatrix} \left( \begin{smallmatrix}
-c_1 + c_2 &\to c_2 \\
-2c_1 + c_3 &\to c_3
\end{smallmatrix}\right) = 0$

\textbf{Bài 17.} Tính các định thức sau

c) $\begin{vmatrix}
0 & b & c & d \\
b & 0 & d & c \\
c & d & 0 & b \\
d & c & b & 0 \\
\end{vmatrix}$

\textbf{Cách 1.} Khai triển Laplace theo cột 1, ta có:

$det(A) = b.A_{2,1} + c.A_{3,1} + d.A_{4,1} = -b\begin{vmatrix}
b & c & d \\
d & 0 & b \\
c & b & 0
\end{vmatrix} + c\begin{vmatrix}
b & c & d \\
0 & d & c \\
c & b & 0
\end{vmatrix} - d\begin{vmatrix}
b & c & d \\
0 & d & c \\
d & 0 & b
\end{vmatrix}$

$= -b\left(c^2b+d^2b-b^3\right) + c\left(c^3-d^2c-b^2c\right) -d\left(b^2d+c^2d-d^3\right) = b^4+c^4+d^4-2b^2c^2-2c^2d^2-2d^2b^2$	

$= b^4+c^4-2b^2c^2-b^2d^2-c^2d^2-2bcd^2-b^2d^2-c^2d^2+2bcd^2+d^4 = \left(b^2-c^2\right)^2-\left(b+c\right)^2d^2-\left(b-c\right)^2d^2+d^4 = \left(b^2-c^2\right)^2d^2 - \left(b^2+c^2\right)^2d^2 - (b-c)^2d^2+d^4 = \lbrack (b+c)^2-d^2 \rbrack \lbrack (b-c)^2-d^2 \rbrack$

$= (b+c-d)(b+c+d)(b-c-d)(b-c+d)$

\textbf{Cách 2.} Đặt $P(x) = \begin{vmatrix}
-x & b & c & d \\
b & -x & d & c \\
c & d & -x & b \\
d & c & b & -x \\
\end{vmatrix}$, thì $P(x)$ là một đa thức bậc 4 có số hạng với lũy thừa cao nhất là $x^4$

$P(x) = \begin{vmatrix}
b+c+d-x & b+c+d-x & b+c+d-x & b+c+d-x \\
b & 0 & d & c \\
c & d & 0 & b \\
d & c & b & 0 \\
\end{vmatrix} (h_2+h_3+h_4+h_1 \to h_1)$

$\Rightarrow P(b+c+d) = 0$



\end{document}